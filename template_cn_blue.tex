%# -*- coding:utf-8 -*-
%% start of file `template_en.tex'.
%% Copyright 2006-1008 Xavier Danaux (xdanaux@gmail.com).
%
% This work may be distributed and/or modified under the
% conditions of the LaTeX Project Public License version 1.3c,
% available at http://www.latex-project.org/lppl/.


\documentclass[11pt,a4paper]{moderncv}

\usepackage{fontspec,xunicode}
\usepackage[slantfont,boldfont]{xeCJK}
\usepackage{xcolor}  % replace by the encoding you are using

\usepackage{ifplatform} % 判断平台


% \setmainfont{Tahoma}
\setmainfont{Liberation Serif}  % 缺省英文字体.serif是有衬线字体sans serif无衬线字体 (WSL使用Liberation Serif替代Times New Roman)

\ifwindows
  % Windows 下使用的代码
  \setCJKmainfont[ItalicFont=KaiTi, BoldFont=SimHei]{SimSun}  % 衬线字体 缺省中文字体为
  \setCJKsansfont{SimSun}
  \setmonofont{FangSong}                 % 等宽字体: FangSong, Consolas
\else
  % Linux/WSL 下使用的代码
  \setCJKmainfont[ItalicFont=AR PL KaitiM GB, BoldFont=WenQuanYi Zen Hei]{AR PL SungtiL GB}  % 衬线字体 缺省中文字体为宋体
  \setCJKsansfont{WenQuanYi Micro Hei}  % 无衬线字体使用文泉驿微米黑
  \setCJKmonofont{WenQuanYi Zen Hei Mono}  % 中文等宽字体
\fi

\ifmacosx
  % macOS 下使用的代码
  \setCJKmainfont[ItalicFont={Kai}, BoldFont={Hei}]{STSong}  % 衬线字体 缺省中文字体为
  \setCJKsansfont{STSong}
  \setCJKmonofont{STFangsong}  % 中文等宽字体
\fi


%-----------------------xeCJK下设置中文字体------------------------------%
% 以下字体设置为 WSL/Linux 兼容版本
\setCJKfamilyfont{song}{AR PL SungtiL GB}  % 宋体 song (使用文鼎宋体)
\newcommand{\song}{\CJKfamily{song}}
% \setCJKfamilyfont{fs}{FangSong_GB2312}  % 仿宋2312 fs (WSL中不可用,已注释)
% \newcommand{\fs}{\CJKfamily{fs}}
\setCJKfamilyfont{yh}{WenQuanYi Micro Hei}  % 微软雅黑 yh (使用文泉驿微米黑替代)
\newcommand{\yh}{\CJKfamily{yh}}
\setCJKfamilyfont{hei}{WenQuanYi Zen Hei}  % 黑体  hei (使用文泉驿正黑)
\newcommand{\hei}{\CJKfamily{hei}}
% \setCJKfamilyfont{hwxh}{STXihei}  % 华文细黑  hwxh (WSL中不可用,已注释)
% \newcommand{\hwxh}{\CJKfamily{hwxh}}
% \setCJKfamilyfont{asong}{Adobe Song Std}  % Adobe 宋体  asong (WSL中不可用,已注释)
% \newcommand{\asong}{\CJKfamily{asong}}
% \setCJKfamilyfont{ahei}{Adobe Heiti Std}  % Adobe 黑体  ahei (WSL中不可用,已注释)
% \newcommand{\ahei}{\CJKfamily{ahei}}
\setCJKfamilyfont{akai}{AR PL KaitiM GB}  % Adobe 楷体  akai (使用文鼎楷体替代)
\newcommand{\akai}{\CJKfamily{akai}}


%------------------------------设置字体大小------------------------%
\newcommand{\chuhao}{\fontsize{42pt}{\baselineskip}\selectfont}  % 初号
\newcommand{\xiaochuhao}{\fontsize{36pt}{\baselineskip}\selectfont}  % 小初号
\newcommand{\yihao}{\fontsize{28pt}{\baselineskip}\selectfont}  % 一号
\newcommand{\erhao}{\fontsize{21pt}{\baselineskip}\selectfont}  % 二号
\newcommand{\xiaoerhao}{\fontsize{18pt}{\baselineskip}\selectfont}  % 小二号
\newcommand{\sanhao}{\fontsize{15.75pt}{\baselineskip}\selectfont}  % 三号
\newcommand{\sihao}{\fontsize{14pt}{\baselineskip}\selectfont}  % 四号
\newcommand{\xiaosihao}{\fontsize{12pt}{\baselineskip}\selectfont}  % 小四号
\newcommand{\wuhao}{\fontsize{10.5pt}{\baselineskip}\selectfont}  % 五号
\newcommand{\subwuhao}{\fontsize{10pt}{\baselineskip}\selectfont}  % 次五号
\newcommand{\xiaowuhao}{\fontsize{9pt}{\baselineskip}\selectfont}  % 小五号
\newcommand{\liuhao}{\fontsize{7.875pt}{\baselineskip}\selectfont}  % 六号
\newcommand{\qihao}{\fontsize{5.25pt}{\baselineskip}\selectfont}  % 七号


% \usepackage{fontawesome}
% \setCJKmainfont[BoldFont={WenQuanYi Micro Hei/Bold}]{WenQuanYi Micro Hei}
% \defaultfontfeatures{Mapping=tex-text}
% \XeTeXlinebreaklocale "zh"
% \XeTeXlinebreakskip = 0pt plus 1pt minus 0.1pt
% moderncv themes
% 使用 letters 图标样式以避免依赖 fontawesome5
\moderncvicons{letters}
\moderncvtheme[blue]{classic}  % optional argument are 'blue' (default), 'orange', 'red', 'green', 'grey' and 'roman' (for roman fonts, instead of sans serif fonts)
% \moderncvtheme[green]{classic}  % idem
% \moderncvtheme[blue,roman]{hht}
% character encoding



% adjust the page margins
\usepackage[scale=0.9]{geometry}
% \setlength{\hintscolumnwidth}{3cm}  % if you want to change the width of the column with the dates
% \AtBeginDocument{\setlength{\maketitlenamewidth}{6cm}}  % only for the classic theme, if you want to change the width of your name placeholder (to leave more space for your address details
\AtBeginDocument{\recomputelengths}  % required when changes are made to page layout lengths

% personal data
\firstname{行}
\familyname{增}
\title{XingZeng}  % optional, remove the line if not wanted
% \address{杭州}{}  % optional, remove the line if not wanted
% \address{上海市青浦区华为练秋湖研发中心}{}  % optional, remove the line if not wanted
\mobile{15209200692}  % optional, remove the line if not wanted
\email{zengxing2025@163.com}  % optional, remove the line if not wanted
% \homepage{Blog: http://geekplux.com}  % optional, remove the line if not wanted
% \social[github]{GitHub: https://github.com/zengxing}
\extrainfo{%
  % LinkedIn: https://cn.linkedin.com/in/zengxing \\
  WeChat: Shuchen\_Xing \\
}

\photo[64pt]{证件照4.jpg}  % '64pt' is the height the picture must be resized to and 'picture' is the name of the picture file; optional, remove the line if not wanted
% \quote{China\TeX 您的LaTeX乐园,TeX\&\LaTeX 王国}  % optional, remove the line if not wante

% \nopagenumbers{}  % uncomment to suppress automatic page numbering for CVs longer than one page


%----------------------------------------------------------------------------------
%            content
%----------------------------------------------------------------------------------
\begin{document}
\maketitle
\vspace*{-14mm}

\section{教育经历}
\cventry{18.09-22.06}{本科}{西北工业大学}{信息安全}{}{}  % arguments 3 to 6 are optional
\cvlistitem{C++/数据结构/操作系统/计算机网络/计算机组成原理/编译原理/数据库}
\cvlistitem{全国大学生信息安全竞赛一等奖}
\cventry{22.09-25.06}{硕士}{西北工业大学}{网络信息安全}{卫星通信安全与智能优化实验室}{}
\cvlistitem{矩阵论/优化理论/深度学习}
  % arguments 3 to 6 are optional


\section{工作经历}
\cventry{25.06--至今}{上海华为技术有限公司}{OpenHarmony方舟编译运行时}{}{方舟运行时团队}{万马奔腾-金码奖}
\cvlistitem{1.独立负责C++公共基础类库的开发与维护工作,参与功能维护、代码优化、漏洞修复、性能提升等工作,在汇编与C++标准库实现层面深入分析解决问题。}
\cvlistitem{2.负责方舟运行时基础库的开发和维护,参与了基础库数值类API接口治理与维护、基础库容器类API接口治理与维护、与社区成员共同维护ArkCompiler开源社区,积极贡献。}
\cvlistitem{3.负责方舟运行时反射与类系统的开发与维护工作,通过语言层与运行时虚拟机层的交互,实现类与对象的动态创建、方法的动态调用、属性的动态获取与设置等。}


\section{项目}
\subsection{科研项目}
\cvline{PKG-eMAE}{物理层密钥生成(PKG)增强的消息加密认证算法(eMAE):基于物理层提供的随机源,对消息进行加密和认证,在低功耗设备上取得了良好的性能和保密性。}
\cvline{PKG in DSN}{在分布式传感网络(DSN)中以PKG效用最大化为目标优化波束聚焦预编码、功率分配与用户配对方案,在多用户、低功耗设备、非正交多址接入、近场通信场景下,实现了更高的总和安全性,实现了物理层安全的目标。}

\subsection{开源项目}
\cvline{ArkCompiler}{方舟编译运行时Built-in基础库维护与开发工作,数值类API接口治理与维护,反射与Class系统开发与维护。}
\cvline{c\_utils}{OpenHarmony体系的公共基础类库,包括IPC通信基础数据结构、智能指针、事件循环、线程池、文件操作等基础类。}
\section{技能}
\cvline{\textbf{语言}}{熟练掌握C++,熟悉STL的使用及实现,熟悉Python/TypeScript/Shell脚本语言;}
\cvline{\textbf{工具}}{熟练使用Git/Vim/GDB等工具;能够在Linux平台中进行C++开发}
\cvline{\textbf{知识}}{熟悉操作系统、计算机网络、计算机组成原理、编译原理、数据库等基础知识,了解LLVM编译器的结构与基本实现;}
\cvline{\textbf{其他}}{能够以英语为工作语言,与海外团队成员进行沟通与协作;阅读与撰写英文技术文档。}

\section{Publications}
\cvline{已录用}{\textbf{Zeng Xing}, Bo Zhao, Bo Xu, et al.  Enhanced Message Encryption Authentication Scheme Based on Physical-Layer Keg Generation in Resource-Limited Internet of Things[J], KSII Transactions on Internet and Information, 2024.}
% \subsection{Vocational}
% \cventry{year--year}{Job title}{Employer}{City}{}{Description}  % arguments 3 to 6 are optional
% \cventry{year--year}{Job title}{Employer}{City}{}{Description}  % arguments 3 to 6 are optional
% \subsection{Miscellaneous}
% \cventry{year--year}{Job title}{Employer}{City}{}{Description line 1\newline{}Description line 2}% arguments 3 to 6 are optional

% \section{Languages}
% \cvlanguage{language 1}{Skill level}{Comment}
% \cvlanguage{language 2}{Skill level}{Comment}
% \cvlanguage{language 3}{Skill level}{Comment}

% \section{Computer skills}
% \cvcomputer{category 1}{XXX, YYY, ZZZ}{category 4}{XXX, YYY, ZZZ}
% \cvcomputer{category 2}{XXX, YYY, ZZZ}{category 5}{XXX, YYY, ZZZ}
% \cvcomputer{category 3}{XXX, YYY, ZZZ}{category 6}{XXX, YYY, ZZZ}

% \section{Interests}
% \cvline{篮球}{\small 体力与技巧}
% \cvline{hobby 2}{\small Description}
% \cvline{hobby 3}{\small Description}

% \renewcommand{\listitemsymbol}{-}  % change the symbol for lists

% \section{Extra 1}
% \cvlistitem{Item 1}
% \cvlistitem{Item 2}
% \cvlistitem[+]{Item 3}  % optional other symbol% XeLaTeX can use any Mac OS X font. See the setromanfont command below.
% Input to XeLaTeX is full Unicode, so Unicode characters can be typed directly into the source.

% The next lines tell TeXShop to typeset with xelatex, and to open and save the source with Unicode encoding.

% !TEX TS-program = xelatex
% !TEX encoding = UTF-8 Unicode

% \section{Extra 2}
% \cvlistdoubleitem[\Neutral]{Item 1}{Item 4}
% \cvlistdoubleitem[\Neutral]{Item 2}{Item 5}
% \cvlistdoubleitem[\Neutral]{Item 3}{}

%% Publications from a BibTeX file
% \nocite{*}
% \bibliographystyle{plain}
% \bibliography{publications}  % 'publications' is the name of a BibTeX file

% \begin{thebibliography}{}
% \bibitem[]{} 移动增强现实可视化综述[C]. ChinaVis 2017.
% \end{thebibliography}


\end{document}


%% end of file `template_en.tex'.

%%% Local Variables:
%%% mode: latex
%%% TeX-command-extra-options: "-shell-escape"
%%% TeX-master: t
%%% TeX-engine: xetex
%%% End:
